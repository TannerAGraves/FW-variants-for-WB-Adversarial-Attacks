\documentclass{article}
\usepackage{graphicx} % Required for inserting images
\usepackage{amsmath}
\usepackage{algorithm}
\usepackage{algpseudocode}
\usepackage{mathrsfs}

\title{Comparison of Frank-Wolfe Varients for White-Box Adversarial Attacks}
\author{Tanner Aaron Graves - 2073559\and Alessandro Pala - 2107800}
\date{June 2024}

\begin{document}

\maketitle

\section{Introduction}
With deep neural networks becoming ubiquitous in application, adversarial attacks have recieved much attention, as it has proved remarkably easy to create adversarial examples- genuine data that undergoes a minimal and unobtrusive corruption process inorder to maximally harm the performance of a model. 
This is typically stated as the following constrained optimization problem:
\begin{equation}
\begin{aligned}
\min_x \quad & f(x)\\
\text{s.t.} \quad & ||x||_p \leq \epsilon
\end{aligned}
\end{equation}
In the case of untargeted attacks on a classifier, where we perturb an example with the aim it be incorrectly predicted as any other class. Here $f(x)$ is the loss function of the attacked model $-\ell(x, \hat{y})$. In the case of targeted attacks we take $f(x) = \ell(x, \hat{y}\neq y)$.
%todo: state if we focus on untarged, or see if I have enough time to implement targeted.
The $L_p$ constraint $||x||_p \leq \epsilon$ directly restricts the size of perturbations made the to the example. An inherient problem of DNNs is often attacks can be consistently sucessful even with very small, even imperceptable $\epsilon$. Different choices of $p$ may be made giving $||x||_p = (\sum_i{x_i^p})^{1/p}$. Or commonly, as we use here $L_\infty(x) = \max_i |x_i|$. 
%Constrained problem
The constrained nature of this problem limits the applicabiliy of a method like gradient descent, and requires integrating knowledge of the constraint space for effective optimization. Methods like Fast Signed Gradient attacks and projected gradient descent are popular choices, but are each limited either by providing maximal and simple adversarial examples or being computationally inefficient. We explore Frank-Wolfe varients wich are well suited to this problem by ensuring feasibility within the constraint set at each iteration with the efficient solving of a Linear Minimization Oracle (LMO). 
% White box vs.black box
Here we focus on white-box adversarial attacks which utilize the model structure for calculating $\nabla f(x)$. In senerios where this is not accessible, attacks are black-box and generally require methods to estimate the gradient with zeroth order methods.

\section{Algorithims}
\subsection{Frank-Wolfe}
\begin{algorithm}
\caption{An algorithm with caption}\label{alg:cap}
\begin{algorithmic}[1]
\Require maximum iterations $T$, stepsizes $\{\gamma_t\}$, convergence tolerence $\delta$
\Ensure $y = x^n$
\State $x_0 = x_{\text{ori}}$
\For{$t = 1,...,T$}
	% for the sets use the notation from FW_varients for consistency
	% M is the condstrained space of the attack and A is the set of vertieces
	\State $s_t = {\arg \min}_{x\in\mathcal{M}} \langle x, \nabla f(x_t)\rangle$ \Comment{LMO step}
	\State $d_t = s_t - x_t$
	\State $x_{t+1} = x_t + \gamma_t d_t$
	% TODO: maybe mention FW gap converg crit here
	\If{$\langle d_t, -\nabla f(x)\rangle < \delta$} \text{return} \hfill \Comment{FW gap convergence criterion}
	\EndIf
\EndFor
\end{algorithmic}
\end{algorithm}

% TODO:explain the LMO, why it is fast, why it stays in the constrained set. How long it takes to compute
Observing oscilation in Frank Wolfe convergence is common and consequence of optimal points lying on a face of $\mathcal{M}$. Since at each iteration the method is moving twrods a vetex of $\mathcal{M}$, which we denote the set of as $\mathcal{S}$, the method "zigzags", moving twords different points in effort to gradually approach the face on which the optimum lies. We implement varients that aim to address this problem to provide better convergece. The simplest of which is adding momentum to standerd frank wolf which replaces the gradient in the LMO calulation in line $(4)$ with a momentum term $m_t  = \beta m_{t-1} + (1-\beta) \nabla f(x_t)$ and initialize $m_0 = \nabla f(x_0)$. By considering this exponentially weighted average of gradient information, momentum varients are emperically observed to have nicer convergence. 
%reference attacks.pdf
% TODO: talk about the LMO. Complexity in FW.pdf
The LMO finds an optimal direction on 
The LMO is also refered tp as the Frank-Wolfe sub problem and can be generally costly to compute, however, in the case of our constraint set $\mathcal{M}$ which is the $L_\infty$ ball, the LMO has a closed form solution:
$$s_t = -\epsilon \text{sign}(\nabla f(x)) +  
% TODO: define duality and convergence criterion as in attacks.pdf, some more details in FW.pdf but less relevant.
The convergence criterion used here is the Frank
\subsection{Away-Step Frank-Wolfe}
\subsection{Pairwise Frank-Wolfe}

\section{Results}
Introduce Datasets
\subsection{Momentum}
\subsection{Early-Stopping (Convergece Criterion)}
It is worth noting that Convergence Criterion For Frank-Wolfe is a somewhat imprecise surrogate for success in the context of adversarial attacks. For many examples, we find that Frank-Wolfe methods create successful attacks several iterations before convergence. We attribute this to an incorrect class probability being grater than the correct class being sufficient for success where convergence is reached when the new output class probability is maximized. 
We observe the convergence of the Frank-Wolfe gap 
\subsection{Stepsize}
The methods for stepsize where implemented: fixed, where $\gamma_t = 1$ for all $t$, linesearching which solves $\arg \min _\gamma f(x + \gamma d_t)$ where $\gamma \in (0,1]$, and decaying stepsize $\gamma_t = \frac{2}{t + 2}$.
\subsection{$\epsilon$ Choice}
Create plot showing how accurate attacks are with different $\epsilon$ constraints.
\section{Convergence Analysis}
The constrained nature of the Adversarial Attack problem means that the norm of the gradient $||\nabla_x f(x)||$ is not a sutible convergence criterion as boundry points need not have $0$ gradient. 
The Frank-Wolfe gap provides provides measure of both optimality and point feasibility. It is a measure of the maximum improvement over the current iteration $x_t$ within the constraints $C$ and defined
$$g(x_t) = \max_{x\in C} \langle x - x_t, -\nabla f(x_t)\rangle$$
We always have $g(x_t) \geq 0$ and its usefullness as a convergence criterion comes from $g(x_t) = 0$ iff $x_t$ is a stationary point. 
% See what the slides say about stationary points
For convex problems, we would have that the linear approximation $f(x_t) + \langle x_t - x, -\nabla f(x_t) \rangle \geq f(x)$. However, the loss of DNNs as commnly the subject of adversarial attacks, are highly non-convex, making this only true locally. This complicate the convergence of Frank-Wolfe in this application, but it is still gaurenteed.

\subsection{Frank-Wolfe}
\subsection{Pairwise Frank-Wolfe}
\subsection{Away-Step Frank-Wolfe}
\end{document}
